\documentclass[11pt]{extreport}

\input{settings}

\begin{document}

% Dla lepszej wydajności można w trakcie pisania zakomentować obydwa include
\input{title_page}
\input{claims_page}
\tableofcontents  %Spis treści

\begingroup
\titleformat{\chapter}[hang]
  {\normalfont\fontsize{16}{20}\bfseries}
  {\thechapter.}{1em}{}
      \clearpage
    \phantomsection

\addcontentsline{toc}{chapter}{Wykaz oznaczeń}      % Opcjonalny wykaz oznaczeń
\chapter*{Wykaz oznaczeń}                           
\endgroup

\chapter{Cel i zakres pracy}              %rozdział

(Styl Standardowy, Times New Roman 11 lub Arial 11, odstęp między liniami tekstu 1.25 wiersza, wcięcie pierwszego wiersza: 0,6). W rozdziale „Cel i zakres pracy” należy przedstawić jasno, co jest przedmiotem pracy. Wyjaśnić cel oraz podać czynności, które zostały wykonane, aby ten cel został osiągnięty. W rozdziale tym można opisać z czego praca będzie się składać, np.: 
Niniejsza praca dyplomowa składać się będzie z dwóch głównych części. Pierwsza z nich poświęcona zostanie omówieniu zagadnień teoretycznych, związanych z wykorzystaniem energii słonecznej, a w szczególności … Druga część tej pracy związana będzie bezpośrednio z wykonywanym projektem…
Jeżeli zostało użyte oprogramowanie przy realizacji pracy należy to oprogramowanie wymienić. Rozdział „Cel i zakres pracy” powinien zając maksymalnie jedną stronę. 

\chapter{Wstęp} 

Tekst pracy należy napisać czcionką Time New Roman 11 lub Arial 11, odstęp 1.25 wiersza, wcięcie pierwszego wiersza: 0,63 (Styl Standardowy). Do całości pracy zastosować wyjustowanie akapitów, lewy margines ma wynosić 3.5 cm natomiast pozostałe marginesy 2.5 cm. Przy drukowaniu pracy należy uwzględni fakt, że jeden z egzemplarzy ma zostać wydrukowany dwustronnie i oprawiony w miękkie oprawki (marginesy lustrzane). Rozdziały główne pracy powinny być umieszczone na następnej (nowej) stronie.

\chapter{Układ graficzny pracy}

Przy pisaniu pracy obowiązuje styl bezosobowy tak jak pokazują to poniższe przykłady: 
Wpływ zachmurzenia na gęstość strumienia promieniowania słonecznego przedstawiono na rysunku 3.1.  – \textbf{poprawnie}
Wpływ zachmurzenia na gęstość strumienia promieniowania słonecznego przedstawia rysunek 3.1.  – \textbf{poprawnie}
Wpływ zachmurzenia na gęstość strumienia promieniowania słonecznego przedstawiłem na rysunku 3.1. – \textbf{niepoprawnie}

\begin{figure}[ht]
    \centering
    \includegraphics[width=0.80\linewidth]{images/Obraz2.png}
    \caption{Wysokość i azymut Słońca dla 52°N \cite{dirac} }
    \label{fig:Obraz2}
\end{figure}
Styl RYS, Times New Roman 11, kursywa, odstęp pomiędzy wierszami pojedynczy. W podpisach rysunków nie dajemy kropki na końcu

\section{Wzory} %podrozdział (h2)

Jeżeli występują w tekście wzory należy je numerować oraz podać opis wielkości w nich występujących. Wzory należy wyśrodkowywać a numerowanie worów wyrównywać do prawej krawędzi tak jak przedstawia to poniższy przykład. Zmienne we wzorach powinny być napisane kursywą. Wzór jest częścią zadania, obowiązują w związku z tym zasady interpunkcji.   

\subsection{Natężenia promieniowania } %podrozdział (h3)

Przykład:
Chwilowa wartość natężenia promieniowania jest parametrem, który wylicza się z zależności:

\begin{equation}
  G'min = U\underset{L}{}*\frac{\Delta T}{\eta\underset{opt}{} } \tag{3.1}\label{eq:3.1} 
\end{equation}

gdzie: \\
$\eta\underset{opt}{}$ - sprawność optyczna kolektora, \\
$U\underset{L}{}$- współczynnik całkowitych strat kolektora, [W/m2K],\\
$\Delta T$ - różnica temperatur pomiędzy czynnikiem solarnym a otoczeniem, [K].\\

\section{Rysunki, tabele i bloki kodu}

Rysunki oraz tabele (tylko dobrej jakości, min. 300dpi) występujące w tekście należy wyśrodkowywać. Każdy rysunek oraz tabela musi zawierać numer oraz podpis. Podpis umieszcza się pod rysunkiem oraz nad tabelą, tak jak to przedstawiają poniższe przykłady. Odstęp pomiędzy tekstem a rysunkiem wynosi jedną linię. \textbf{Uwaga!} Nie można wstawiać do pracy rysunków oraz tabel bez ich opisu/komentarza w tekście. Rysunek lub tabela pozbawione komentarzy w tekście są bezwartościowe i obniżają jakość pracy. 
Przykład: Wysokość i azymut Słońca we wszystkich porach roku przedstawia rysunek 2.1. Gęstość strumienia promieniowania zależy od zachmurzenia co ilustruje rys.2.2. \\

\begin{figure}[ht]
    \centering
    \includegraphics[width=0.75\linewidth]{images/Obraz3.png}
    \caption{Wpływ zachmurzenia na gęstość strumienia promieniowania }
    \label{fig:Obraz3}
\end{figure}

Wstawiane do pracy rysunki, schematy, wykresy, zdjęcia są traktowane jak rysunki i nie rozróżnia się ich w podpisach. Pomimo tego, że wstawiany jest wykres podpisywany jest on jako kolejny rysunek. Rysunki umieszcza się w pracy przez wstawienie z pliku *.jpg.
Numeracje rysunków, tabel oraz wzorów przeprowadza się w obrębie głównych rozdziałów. Wyniki obliczeń dla poszczególnych miesięcy przedstawia tabela 2.1.

\subsection{Tabela manualna}

\begin{table}[ht]
\centering			          %justowanie całej tabeli na środku strony
\caption{Przykładowa tabela}  %opis tabeli
\label{tab:przyklad}		  %etykieta, przyda się do odwołań
\begin{tabular}{|c|c|c|}      %justowanie c-enter, l-eft, r-ight
\hline					      %pozioma linia
Dane 1 & Dane 2 & Dane 3 \\
\hline                                 % linia pozioma
Komórka 1 & Komórka 2 & Komórka 3 \\  % & oddziela kolumny
\hline
Komórka 4 & Komórka 5 & Komórka 6 \\  % \\ zamyka kolumnę
\hline
\end{tabular}
\end{table}

\subsection{Tabela z pliku csv}
To Do

\subsection{Blok kodem programu }
Można również bezpośrednio z pliku % \inputminted{python}{filename.py}
\begin{minted}{python}

def f(x):
    return 1/(1 + x**2)

# Implementing trapezoidal method
def trapezoidal(x0,xn,n):
    # calculating step size
    h = (xn - x0) / n
    
    # Finding sum 
    integration = f(x0) + f(xn)
    
    for i in range(1,n):
        k = x0 + i*h
        integration = integration + 2 * f(k)
    
    # Finding final integration value
    integration = integration * h/2
    
    return integration
\end{minted}

\section{Cytowany tekst }

W przypadku cytowania fragmentów tekstów, tabel, rysunków lub wzorów z literatury należy zaznaczyć autora tego cytowania w tekście pracy.

\textbf{Sposób pierwszy:} Cytowanie poprzez umieszczenie odpowiedniego numeru w nawiasie kwadratowym. Powołanie na literaturę umieszcza się po cytowanym fragmencie teksu przed kropką np.: Na skutek procesów zachodzących w atmosferze, do powierzchni Ziemi dociera jedynie 39-45\% promieniowania pozaatmosferycznego w skali roku [10].  - \textbf{poprawnie}    
Na skutek procesów zachodzących w atmosferze, do powierzchni Ziemi dociera jedynie 39-45\% promieniowania pozaatmosferycznego w skali roku. [10]  - \textbf{niepoprawnie }

\textbf{Sposób drugi:} Cytowanie poprzez powołanie się na autorów cytowanej pracy oraz roku publikacji pracy np.: Nowak i Kowalski (2009) w swojej pracy prezentują … 
Jeżeli autorami pracy są więcej niż dwie osoby należy użyć w cytowaniu tylko nazwiska pierwszego autora np.: Nowak i inni (2010) w swojej pracy prezentują …
Jeżeli ci sami autorzy (lub autor) w jednym roku wydali kilka publikacji, które są w pracy cytowane należy po roku cytowanej publikacji dodać jeszcze literę zgodnie z kolejnością umieszczenia publikacji w wykazie literatury np.: Nowak (2011a) oraz Nowak (2011c) w swoich pracach przedstawia …

\chapter{Wnioski}

We wnioskach należy w przejrzysty sposób podsumować pracę, napisać czy założony cel pracy został osiągnięty i w jakim stopniu. Jeżeli praca ma charakter projektu, autor powinien podać zalecenia projektowe wynikające z przeprowadzonych obliczeń/analiz. Jeżeli z pracy wynikają wnioski przyszłościowe należy je wymienić. Należy użyć stylu, rozmiaru czcionki i odstępu takich jak w całej pracy.


\printbibliography[title={Literatura}]

\begingroup
\titleformat{\chapter}[hang]
  {\normalfont\fontsize{16}{20}\bfseries\centering}
  {\thechapter.}{1em}{}
      \clearpage
    \phantomsection

\addcontentsline{toc}{chapter}{Summary}      % Opcjonalny wykaz oznaczeń
\chapter*{Summary}                           
\endgroup

W tym miejscu należy zamieścić streszczenie pracy w języku angielskim o objętości min. 2500 znaków ze spacjami.

\end{document}


